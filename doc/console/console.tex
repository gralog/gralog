\section{Working with the console}

The console is a tool to work with graphs by entering commands rather
than by clicking the mouse buttons. This is especially convenient if
you want to create large subgraphs with a regular pattern.

By default, the console is the frame below the main frame with the
graph. Here, in the bottom line, you can enter commands just like in a
terminal.

\subsection{Lists}

The objects you work with are \emph{lists:} lists of vertices and
lists of edges. We concentrate on vertex lists in this manual, edge
lists behave in the same way. All lists can be created out of the
default list of all vertices and the internal list of all currently
selected vertices. The latter can be made with the mouse in the
obvious way or you can write in the command line:
\begin{tcolorbox}
  select all vertices,
\end{tcolorbox}
\noindent
to select all vertices. By the way, in a similar way, all vertices are
deselected:
\begin{tcolorbox}
deselect all vertices.
\end{tcolorbox}

Now use the selection to create a new list. The corresponding command
is \texttt{filter}: one filters the elements from an existing list
according to certain criteria and saves them into another list. If the
other list already exists, the new elements are appended to it (even
if they are already contained in the target list). 

\begin{tcolorbox}
  filter all selected where no condition to VL1
\end{tcolorbox}

\subsection{Filter}

Now we created a new list \texttt{VL1} containing all vertices of the
graph. Let us now create a new list that contains only vertices with a
certain property, say those whose colour is blue. We can now use
\texttt{VL1} as a source of vertices.

\begin{tcolorbox}
  filter VL1 where fill color blue to VLblue
\end{tcolorbox}

Now the list \texttt{VLblue} contains all vertices with fill color
\texttt{blue}. Ne create a new list containing all vertices whose fill
color is red. Note in the following example that instead of the
keyword \texttt{where} one can write \texttt{such that} or
\texttt{st}. Note also that instead of \texttt{fill color} it is
possible to use just \texttt{fill}:

\begin{tcolorbox}
  filter VL1 st fill red to VLred
\end{tcolorbox}

The full specification of the filter command is as follows. Note that
Gralog does not distinguish between the upper and the lower cases.


\begin{tcolorbox}
  FILTER <what> WHERE|ST|(SUCH THAT) <parameterS> TO <list identifier>
  [ignored trash]
\end{tcolorbox}
 
\noindent where

\begin{tcolorbox}
  
  \begin{tabular}[H]{rl}
    <what> := &(ALL VERTICES) | (ALL EDGES)\\
              &| ([ALL ]?SELECTED VERTICES) | ([ALL ]?SELECTED EDGES)\\
              & | <list id> 
  \end{tabular}
and \texttt{<list>} can be a list of vertices or a list of edges;
\begin{tabular}[H]{rl}
  <parameterS> := &<parameterS> <parameterS> | <parameter> <value> \\
                  & | <bool\_parameter> | formula <integer expression
                    (i)>\\
  \end{tabular}
  and
  \begin{tabular}[H]{rl}
  <parameter> := &NO CONDITION | FILL|(FILL COLOR) | STROKE|(STROKE
    COLOR) \\
                 &| THICKNESS | WIDTH | HEIGHT | SIZE | ID\\
                 &| SHAPE | WEIGHT | TYPE | (EDGE TYPE)|EDGETYPE \\
                 &| DEGREE | INDEGREE | OUTDEGREE\\
                 &| BUTTERFLY // not implemented yet
  \end{tabular}
  and finally
  \begin{tabular}[H]{rl}
  <bool\_parameter> := &HAS SELFLOOP | DIRECTED | HAS LABEL
  \end{tabular}
\end{tcolorbox}

%%% Local Variables:
%%% mode: latex
%%% TeX-master: "main"
%%% End:
